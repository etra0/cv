%% start of file `template.tex'.
%% Copyright 2006-2013 Xavier Danaux (xdanaux@gmail.com).
%
% This work may be distributed and/or modified under the
% conditions of the LaTeX Project Public License version 1.3c,
% available at http://www.latex-project.org/lppl/.


\documentclass[11pt,a4paper,sans]{moderncv}        % possible options include font size ('10pt', '11pt' and '12pt'), paper size ('a4paper', 'letterpaper', 'a5paper', 'legalpaper', 'executivepaper' and 'landscape') and font family ('sans' and 'roman')

% moderncv themes
\moderncvstyle{classic}                             % style options are 'casual' (default), 'classic', 'oldstyle' and 'banking'
\moderncvcolor{orange}                               % color options 'blue' (default), 'orange', 'green', 'red', 'purple', 'grey' and 'black'
%\renewcommand{\familydefault}{\sfdefault}         % to set the default font; use '\sfdefault' for the default sans serif font, '\rmdefault' for the default roman one, or any tex font name
%\nopagenumbers{}                                  % uncomment to suppress automatic page numbering for CVs longer than one page

% character encoding
\usepackage[utf8]{inputenc}                       % if you are not using xelatex ou lualatex, replace by the encoding you are using
%\usepackage{CJKutf8}                              % if you need to use CJK to typeset your resume in Chinese, Japanese or Korean

% adjust the page margins
\usepackage[scale=0.75]{geometry}
%\setlength{\hintscolumnwidth}{3cm}                % if you want to change the width of the column with the dates
%\setlength{\makecvtitlenamewidth}{10cm}           % for the 'classic' style, if you want to force the width allocated to your name and avoid line breaks. be careful though, the length is normally calculated to avoid any overlap with your personal info; use this at your own typographical risks...

% personal data
\name{Sebastián}{Aedo}
%\title{Resumé title}                               % optional, remove / comment the line if not wanted
\address{Valle Hermoso, 2712}{Los Pinos}{Quilpué, V Región}% optional, remove / comment the line if not wanted; the "postcode city" and and "country" arguments can be omitted or provided empty
\phone[mobile]{+56~(9)~5197~4866}                   % optional, remove / comment the line if not wanted
%\phone[fixed]{+2~(345)~678~901}                    % optional, remove / comment the line if not wanted
%\phone[fax]{+3~(456)~789~012}                      % optional, remove / comment the line if not wanted
\email{sebastian.aedo29@gmail.com}                               % optional, remove / comment the line if not wanted
\homepage{saedo.me}                         % optional, remove / comment the line if not wanted
%\extrainfo{additional information}                 % optional, remove / comment the line if not wanted
%\photo[64pt][0.4pt]{picture}                       % optional, remove / comment the line if not wanted; '64pt' is the height the picture must be resized to, 0.4pt is the thickness of the frame around it (put it to 0pt for no frame) and 'picture' is the name of the picture file
%\quote{Some quote}                                 % optional, remove / comment the line if not wanted

% to show numerical labels in the bibliography (default is to show no labels); only useful if you make citations in your resume
%\makeatletter
%\renewcommand*{\bibliographyitemlabel}{\@biblabel{\arabic{enumiv}}}
%\makeatother
%\renewcommand*{\bibliographyitemlabel}{[\arabic{enumiv}]}% CONSIDER REPLACING THE ABOVE BY THIS

% bibliography with mutiple entries
%\usepackage{multibib}
%\newcites{book,misc}{{Books},{Others}}
%----------------------------------------------------------------------------------
%            content
%----------------------------------------------------------------------------------
\begin{document}
%\begin{CJK*}{UTF8}{gbsn}                          % to typeset your resume in Chinese using CJK
%-----       resume       ---------------------------------------------------------
\makecvtitle

\section{Educación}
\cventry{2011--2014}{Enseñanza Media}{Colegio Rubén Castro}{Viña del Mar}{}{}  % arguments 3 to 6 can be left empty
\cventry{2015--2017}{Ingeniería Civil Informática}{Universidad Técnica Federico Santa María}{Valparaíso}{Pregrado}{Actualmente cursando tercer año en mis estudios.}
\subsection{Otros}
\cventry{2014}{Curso de Programación en C}{Pontificia Universidad Católica de Valparaíso}{}{}{Fue mi primer acercamiento a la programación. En este curso se dió un conocimiento básico de programación.}

% \section{Master thesis}
% \cvitem{title}{\emph{Title}}
% \cvitem{supervisors}{Supervisors}
% \cvitem{description}{Short thesis abstract}

\section{Experiencia}
%\subsection{Vocational}
\cventry{2016--2017}{Ayudante del laboratorio LabComp}{Departamento de Informática}{}{}
{
Durante mi segundo año en la universidad, decidí postular al laboratorio de computación del departamento de informática, más conocido como \emph{LabComp}. En este laboratorio nos encargamos de mantener servicios para la comunidad informática; algunos de estos son \textbf{Conexión remota, GitLab, Zimbra}. Trabajamos principalmente con sistemas \emph{GNU/Linux}, en \emph{Fedora} y \emph{CentOS}.
}%
%
\cventry{2016--2017}{Ayudante del Campamento  STEM}{Universidad Técnica Federico Santa María}{Valparaíso}{}
{
Participé dos veces como ayudante del Campamento STEM, organizado por la Universidad Técnica Federico Santa María, dónde se les daba un primer acercamiento a la programación a estudiantes de enseñanza media. En el curso se enseñaba \emph{Python} junto con una breve introducción al módulo \emph{PyGame}.
}

\subsection{Otros}
\cventry{2016--2017}{Participante Hackathon}{Facebook}{Santiago, Chile}{}
{
Participé en las primeras dos hackathones que ha realizado la empresa \emph{Facebook} en Chile, teniendo la oportunidad de compartir con ingenieros
del rubro quienes me dieron consejos acerca de la industria. En la primera iteración desarrollamos una aplicación móvil en Android, y en la segunda una Aplicación Web con el framework \emph{Django}.
}
\cventry{2017}{Asistente de venta}{Bellsport}{Viña del Mar, Chile}{}
{Trabajé como apoyo de venta los meses de Diciembre, Enero y Febrero en la empresa Bellsport.}

\section{Habilidades de Computación}
\cvitem{\textbf{Python}}{Uno de los lenguajes que más domino. Lo he ocupado tanto como para scripting como para proyectos enfocados a la comunidad. He trabajado con el framework \emph{Django} para hacer web-apps, \emph{PyGame} para hacer pequeños \emph{sketchs} de videojuegos, y últimamente muchos módulos enfocados en data science.}
%
\cvitem{\textbf{JavaScript}}{Lenguaje que he ocupado tanto para Front-End, como Back-End. Para Front-End manejo principalmente \emph{JQuery}, y actualmente aprendiendo \emph{D3JS}, framework enfocado en la visualización de datos. En el caso de Back-End, he trabajado en proyectos donde se utiliza \emph{NodeJS, ExpressJS} junto a \emph{MySQL}.}%

\cvitem{\textbf{C++}}{Conocimiento de las estructuras de datos, punteros y orientación a objetos. Lo utilicé principalmente en un campamento de programación competitiva realizado el año 2016.}%

\cvitem{\textbf{Administrador de Sistemas}}{He trabajado con sistemas \emph{GNU/Linux}, principalmente con \emph{Fedora} y \emph{CentOS}. Conozco el funcionamiento de firewalls, automatización de formateos con kickstarts y mantenimiento de sistemas. Conocimiento intermedio de sistemas Unix-like.}
\cvitem{\textbf{Otros}}{Manejo de herramientas como \textbf{Git, Latex, Excel}, y scripting en \emph{Bash}. Me dedico a automatizar varios procesos en mi computador.}


\section{Intereses/Hobbies}
\cvitem{\textbf{Programación}}{Mi principal interés es todo lo relacionado con el Software--Developement. La mayor parte de mi tiempo libre la dedico a crear pequeños proyectos con el fin de estar en un constante aprendizaje. Alguno de los proyectos más relevantes son el mantenimiento de un Bot en Telegram, que recibe alrededor de 15 consultas diariamente, que fue enfocado a la comunidad Sansana. Otro proyecto relevante fue la creación de una malla interactiva para la malla Informática, proyecto por el cuál abogaré para que entre en la página oficial del departamento de Informática. Además, mantengo un pequeño {\color{color1}\href{http://saedo.me}{blog}} donde hago experimentos relacionados con la visualización de datos y la exploración de éstos.}
\cvitem{\textbf{Fotografía}}{Me gusta mucho tomar fotografías en mi tiempo libre, principalmente retratos.}
\cvitem{\textbf{Viajar}}{Me gusta la idea de ir explorando lugares al aire libre, es por esto que poseo una motocicleta la cual me permite conocer lugares de forma más cercana.}

\section{Objetivos/Metas}
\cvitem{}{Mi objetivo es realizar una práctica laboral la cuál me ayude a acercarme más a la industria y tener un conocimiento más profundo en las tecnologías actuales. Me considero una persona que le encanta estar en constante aprendizaje, en pos de crecer y satisfacer necesidades de la comunidad.}

% \section{References}
% \begin{cvcolumns}
%   \cvcolumn{Category 1}{\begin{itemize}\item Person 1\item Person 2\item Person 3\end{itemize}}
%   \cvcolumn{Category 2}{Amongst others:\begin{itemize}\item Person 1, and\item Person 2\end{itemize}(more upon request)}
%   \cvcolumn[0.5]{All the rest \& some more}{\textit{That} person, and \textbf{those} also (all available upon request).}
% \end{cvcolumns}

% Publications from a BibTeX file without multibib
%  for numerical labels: \renewcommand{\bibliographyitemlabel}{\@biblabel{\arabic{enumiv}}}% CONSIDER MERGING WITH PREAMBLE PART
%  to redefine the heading string ("Publications"): \renewcommand{\refname}{Articles}
% \nocite{*}
% \bibliographystyle{plain}
% \bibliography{publications}                        % 'publications' is the name of a BibTeX file

% % Publications from a BibTeX file using the multibib package
% %\section{Publications}
% %\nocitebook{book1,book2}
% %\bibliographystylebook{plain}
% %\bibliographybook{publications}                   % 'publications' is the name of a BibTeX file
% %\nocitemisc{misc1,misc2,misc3}
% %\bibliographystylemisc{plain}
% %\bibliographymisc{publications}                   % 'publications' is the name of a BibTeX file

% \clearpage
% %-----       letter       ---------------------------------------------------------
% % recipient data
% \recipient{Company Recruitment team}{Company, Inc.\\123 somestreet\\some city}
% \date{January 01, 1984}
% \opening{Dear Sir or Madam,}
% \closing{Yours faithfully,}
% \enclosure[Attached]{curriculum vit\ae{}}          % use an optional argument to use a string other than "Enclosure", or redefine \enclname
% \makelettertitle

% Lorem ipsum dolor sit amet, consectetur adipiscing elit. Duis ullamcorper neque sit amet lectus facilisis sed luctus nisl iaculis. Vivamus at neque arcu, sed tempor quam. Curabitur pharetra tincidunt tincidunt. Morbi volutpat feugiat mauris, quis tempor neque vehicula volutpat. Duis tristique justo vel massa fermentum accumsan. Mauris ante elit, feugiat vestibulum tempor eget, eleifend ac ipsum. Donec scelerisque lobortis ipsum eu vestibulum. Pellentesque vel massa at felis accumsan rhoncus.

% Suspendisse commodo, massa eu congue tincidunt, elit mauris pellentesque orci, cursus tempor odio nisl euismod augue. Aliquam adipiscing nibh ut odio sodales et pulvinar tortor laoreet. Mauris a accumsan ligula. Class aptent taciti sociosqu ad litora torquent per conubia nostra, per inceptos himenaeos. Suspendisse vulputate sem vehicula ipsum varius nec tempus dui dapibus. Phasellus et est urna, ut auctor erat. Sed tincidunt odio id odio aliquam mattis. Donec sapien nulla, feugiat eget adipiscing sit amet, lacinia ut dolor. Phasellus tincidunt, leo a fringilla consectetur, felis diam aliquam urna, vitae aliquet lectus orci nec velit. Vivamus dapibus varius blandit.

% Duis sit amet magna ante, at sodales diam. Aenean consectetur porta risus et sagittis. Ut interdum, enim varius pellentesque tincidunt, magna libero sodales tortor, ut fermentum nunc metus a ante. Vivamus odio leo, tincidunt eu luctus ut, sollicitudin sit amet metus. Nunc sed orci lectus. Ut sodales magna sed velit volutpat sit amet pulvinar diam venenatis.

% Albert Einstein discovered that $e=mc^2$ in 1905.

% \[ e=\lim_{n \to \infty} \left(1+\frac{1}{n}\right)^n \]

% \makeletterclosing

%\clearpage\end{CJK*}                              % if you are typesetting your resume in Chinese using CJK; the \clearpage is required for fancyhdr to work correctly with CJK, though it kills the page numbering by making \lastpage undefined
\end{document}


%% end of file `template.tex'.
